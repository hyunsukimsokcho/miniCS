\documentclass[sigchi]{acmart}
\usepackage{graphicx}

\begin{document}

\title{MiniCS: Critical Section Minimisation in Concurrent Programming}
\author{Hyunsu Kim}
\affiliation{
  \institution{KAIST}
  \city{Daejeon, Rep. of Korea}
}
\email{hyunsu.kim00@kaist.ac.kr}

\maketitle

\begin{abstract}
As the era of artificial intelligence, machines consume massive amount of data and drastic amount of computations. Most programmers often feel fear from multi-threaded code due to its nondeterministic behavior. In order to tame the parallelism with scalability, using synchoronization primitives such as lock and semaphore is necessary to avoid race conditions. However, it is usually known that there is no gold standard for setting critical sections. (i.e. when to acquire lock, then when to release the lock) In this work, we tackle such intractability by applying heuristics.
\end{abstract}

\section{Introduction}
Here is the text of your introduction.

\subsection{Subsection Heading Here}
Write your subsection text here.

\section{Conclusion}
Write your conclusion here.

\end{document}
