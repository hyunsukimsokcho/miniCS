\section{background}

Minimising critical section is one of the most important aspect when refactoring multi-threaded code. However, it is impossible to compare the performance of any two distinct version of code by just measuring the execution time. It is hard to guarantee those measured durations solely depend on each code execution, unless we take full control over thread scheduling, which is totally up to operating systems. In order to overcome such intractability, we took approach counting machine instructions especially in Intel x86 architecture. Further details about how our evaluations went through will be mentioned in section 3 and 4. Here, we briefly go over two concepts that needs to be explained before moving on to details.

\subsection{Data race\label{sec:data_race}}

A {\it{data race}} is a situation where multiple threads try to access shared data at once, and the result depends on the execution order of the threads. For example, consider the following code:

\begin{lstlisting}[frame=tb, xleftmargin=2em, framexleftmargin=1.5em, numbers=left]
int a = 0;
int foo() {
    a++;
}
\end{lstlisting}

Suppose two threads call the \verb|foo| function at once. If the threads do not run concurrently, then the value of \verb|a| will be 2. However, this is not always the case when there is concurrency, because \verb|a++| actually consists of three steps: reading the value, incrementing the value, and writing back the value. The following sequence of actions may occur as a result: \begin{enumerate}
    \item Thread 1 reads the value of \verb|a|, which is 0.
    \item Thread 2 reads the value of \verb|a|, which is 0.
    \item Thread 1 increments its value, making it 1.
    \item Thread 2 increments its value, making it 1.
    \item Thread 1 writes 1 into the value of \verb|a|.
    \item Thread 2 writes 1 into the value of \verb|a|.
\end{enumerate}

We use locks to prevent this situation. When a thread tries to acquire a lock, but another thread owns the lock, then it must wait before the other thread releases the lock. For example, in the modified code below, two threads would wait for each other to read the value of verb|a|, increment it, and write it back. This keeps the global variable from being accessed twice at the same time.

\begin{lstlisting}[frame=tb, xleftmargin=2em, framexleftmargin=1.5em, numbers=left]
int a = 0;
int foo() {
    pthread_mutex_lock(&mutex_lock);
    a++;
    pthread_mutex_unlock(&mutex_lock);
}
\end{lstlisting}

\subsection{Machine instructions}
In compiled languages like \textit{C/C++}, each line of code is translated into multiple machine instructions. For example, consider \textbf{\textit{a++}} at third line of code snippet in \ref{sec:data_race}. It is adding one to integer variable \textbf{\textit{a}}. Although it is a single line of code in \textit{C}, a machine equipped with Intel x86 processors translates it into following instructions:
\begin{itemize}
	\item[] \textbf{ld} \,\,\, \$r1, \textit{a}
	\item[] \textbf{add} \$r1, \$r1, 1
	\item[] \textbf{st} \,\,\, \$r1, \textit{a}
\end{itemize}

It first loads the value of \textbf{\textit{a}} from memory into register \textit{\$r1}. Then, add one to the register value. Finally, it stores the result back to the memory location of \textbf{\textit{a}}. Note that these machine instructions are the commit point for each computation within a machine. It means that we can ensure there is no thread interleaving while executing each instruction. In fact, this is the reason why we are safe to count the number of instructions to measure performance of target program independent from other programs being executed.