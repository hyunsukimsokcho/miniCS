\subsection{Population generation}

As a prototype, we assume the following constraints:

\begin{enumerate}
    \item All scopes are explicitly given with curly braces.
    \item In the \verb|if|-branches, \verb|for|-branches, and \verb|while|-branches, there is no line break between the closing parenthesis and the opening brace. There is no line break between the braces and \verb|else|.
\end{enumerate}

Under these assumptions, we find a \textit{valid range} $[a, b]$, and modify the code so that a lock is held at the end of the $a$-th line and released at the end of the $b$-th line. A valid range to lock must satisfy the following four conditions:

\begin{enumerate}
    \item The range must be contained in a function.
    \item There must be a reference to a global variable inside the range. Otherwise there would be no need to place a lock.
    \item Every scope must either fully contain a range, or be fully contained in a range. This is to prevent a thread from holding its own lock or releasing a lock that it does not hold.
    \item There should be no \verb|return| statement in the range. This is to prevent a thread from finishing while holding a lock.
\end{enumerate}

For example, consider the following code:


\begin{lstlisting}[frame=tb, xleftmargin=2em, framexleftmargin=1.5em, numbers=left]
int arr[100];
int bar(int n) {
    int x = 0;
    for (int i = 0; i <= n; ++i) {
        arr[i] = x;
        x+= i;
    }
    return arr[n];
}
\end{lstlisting}

A range $[4, 5]$ is valid; in this case, we hold and release a lock right before and after executing \verb|arr[i] = x|. On the other hand, $[1, 9]$ is invalid because it is not contained in a function. $[5, 6]$ is invalid because there are no global variables. $[2, 5]$ is invalid because the scope inside the \verb|for| loop neither contains $[2, 5]$ nor is contained in $[2, 5]$. Finally, $[7, 8]$ is invalid because it contains \verb|return|.

We use Clang AST to detect the references to global variables. An expression in a function refers to a global variable if there is a \verb|DeclRefExpr| node whose reference variable has a global storage:

\begin{lstlisting}[frame=tb, xleftmargin=2em, framexleftmargin=1.5em, numbers=left]
static StatementMatcher global_match =
  declRefExpr(
    to(
      varDecl(
        hasGlobalStorage()
      ).bind("globalVar")
    ),
    hasAncestor(
      functionDecl().bind("function")
    )
  ).bind("globalRef");
  \end{lstlisting}

Next, a scope in a function is detected by a \verb|CoumpoundStmt| node:

\begin{lstlisting}[frame=tb, xleftmargin=2em, framexleftmargin=1.5em, numbers=left]
static StatementMatcher scope_match =
  compoundStmt(
    hasAncestor(
      functionDecl().bind("function")
    )
  ).bind("scope");
\end{lstlisting}

Finally, a return statement in a function is detected by a \verb|ReturnStmt| node:

\begin{lstlisting}[frame=tb, xleftmargin=2em, framexleftmargin=1.5em, numbers=left]
static StatementMatcher return_match =
  returnStmt(
    hasAncestor(
      functionDecl().bind("function")
    )
  ).bind("return");
\end{lstlisting}

After finding all references to global variables, all return statements, and all scopes, we output all possible valid ranges in the program. To generate a population, we choose zero or more valid ranges such that no two ranges with the same kind of lock do not intersect.
