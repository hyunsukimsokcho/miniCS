\subsection{Population generation}

To keep the programs simple, we assume the following constraints:

\begin{enumerate}
    \item There are no comments, built-in functions, (TODO)
    \item All scopes are explicitly given with curly braces. (TODO write about line breaks)
\end{enumerate}

Under these assumptions, we find a \textit{valid range} $[a, b]$, and modify the code so that a lock is held at the end of the $a$-th line and released at the end of the $b$-th line. A valid range to lock must satisfy the following four conditions:

\begin{enumerate}
    \item The range must be contained in a function.
    \item There must be a reference to a global variable inside the range. Otherwise there would be no need to place a lock.
    \item Every scope must either fully contain a range, or be fully contained in a range. This is to prevent a thread from holding its own lock or releasing a lock that it does not hold.
    \item There should be no \verb|return| statement in the range. This is to prevent a thread from finishing while holding a lock.
\end{enumerate}

For example, consider the following code:


\begin{lstlisting}[frame=tb, xleftmargin=2em, framexleftmargin=1.5em, numbers=left]
int arr[100];
int bar(int n)
{
    int x = 0;
    for (int i = 0; i <= n; ++i)
    {
        arr[i] = x;
        x+= i;
    }
    return arr[n];
}
\end{lstlisting}

A range $[6, 7]$ is valid; in this case, we hold and release a lock right before and after executing \verb|arr[i] = i|. On the other hand, $[0, 1]$ (Holding a lock before the first line) is invalid because it is not contained in a function. $[7, 8]$ is invalid because there are no global variables. $[3, 7]$ is invalid because the scope inside the \verb|for| loop neither contains $[3, 7]$ nor is contained in $[3, 7]$. Finally, $[9, 10]$ is invalid because it contains \verb|return|.

We used Clang AST to detect the references to global variables. An expression in a function refers to a global variable if there is a \verb|DeclRefExpr| node whose reference variable has a global storage:

\begin{lstlisting}[frame=tb, xleftmargin=2em, framexleftmargin=1.5em, numbers=left]
static StatementMatcher global_match =
  declRefExpr(
    to(
      varDecl(
        hasGlobalStorage()
      ).bind("globalVar")
    ),
    hasAncestor(
      functionDecl().bind("function")
    )
  ).bind("globalRef");
  \end{lstlisting}


Next, a scope in a function is detected by a \verb|CoumpoundStmt| node:

\begin{lstlisting}[frame=tb, xleftmargin=2em, framexleftmargin=1.5em, numbers=left]
static StatementMatcher scope_match =
  compoundStmt(
    hasAncestor(
      functionDecl().bind("function")
    )
  ).bind("scope");
  \end{lstlisting}


After finding all references to global variables and all scopes, we output all possible valid ranges in the program. To generate a population, we choose zero or more valid ranges such that no two ranges with the same kind of lock do not intersect.