\documentclass[sigconf]{acmart}
\settopmatter{printacmref=false} % Removes citation information below abstract
\renewcommand\footnotetextcopyrightpermission[1]{} % removes footnote with conference information in first column

\usepackage{graphicx}
\usepackage{comment}
\usepackage{float}
\usepackage{kotex}
\usepackage{listings} 
\lstset{language=C++}

\usepackage{hyperref}
\hypersetup{
    colorlinks=true,
    linkcolor=blue,
    filecolor=magenta,      
    urlcolor=cyan,
}


\copyrightyear{2019}
\acmYear{2019}
\acmConference{CS454 AI Based Software Engineering}{2019 Fall}{KAIST}
\setcopyright{none}
\acmDOI{}

\begin{document}

\title{MiniCS: Critical Section Minimisation in Concurrent Programming}
\author{Hyunsu Kim}
\affiliation{
  \institution{KAIST}
  \city{Daejeon, Rep. of Korea}
}
\email{hyunsu.kim00@kaist.ac.kr}

\author{Jaemin Yu}
\affiliation{
  \institution{KAIST}
  \city{Daejeon, Rep. of Korea}
}
\email{platinant@kaist.ac.kr}

\author{Doam Lee}
\affiliation{
  \institution{KAIST}
  \city{Daejeon, Rep. of Korea}
}
\email{ehdkacjswo@kaist.ac.kr}

\author{Jaemin Choi}
\affiliation{
  \institution{KAIST}
  \city{Daejeon, Rep. of Korea}
}
\email{jmchoi98@kaist.ac.kr}

\author{Heeju Wi}
\affiliation{
  \institution{KAIST}
  \city{Daejeon, Rep. of Korea}
}
\email{bb0711@kaist.ac.kr}

\begin{abstract}
Modern machines consume massive amount of data as well as drastic amount of computations, as the era of artificial intelligence. In the hardware manufacturing industry, developing SoC optimized to such computations is an already on-going process. Multi-core with machines are prevalent nowadays. Still, thanks to the end of Moore's law and Dennard's scaling, software implementation that synchronizes multiple threads is unavoidable in order to achieve reasonably scalable computing. However, it is often a fear to write multi-threaded code due to its non-deterministic behavior. In order to tame the parallelism with scalability, using synchronization primitives such as lock and semaphore is mandatory to prevent race conditions. While which API to use is rather trivial, it is usually known that setting critical sections (when to acquire a lock, then when to release the lock) is not trivial at all. In this work, we tackle such intractability by applying combination of heuristics for search and static analysis for rigorous evaluation and correctness.
\end{abstract}

\maketitle

\section{Introduction}
Multi-threaded codes often come up with the problem called data race. It means multiple threads access the same memory location concurrently and at least one of them is writing. It may cause computation give wrong output depending on their execution order. It can be resolved by using exclusive locks to control accesses to that memory. But abusing lock will interrupt other threads' execution and it will lead to longer execution time.
So it's important to protect shared memory location by using locks and also minimizing the interval(execution time) between lock and unlock. But it's hard to be achieved by hand, and some of races can be only detected on runtime.
Our project MiniCS is solution for this problem. It produces population of candidate codes with locks and perform GA to get best code that's without data race and has minimum "lock interval" (Lock interval means "execution time between lock including lock and unlock themselves" on this paper).

\section{background}

Minimising critical section is one of the most important aspect when refactoring multi-threaded code. However, it is impossible to compare the performance of any two distinct version of code by just measuring the execution time. It is hard to guarantee those measured durations solely depend on each code execution, unless we take full control over thread scheduling, which is totally up to operating systems. In order to overcome such intractability, we took approach counting machine instructions especially in Intel x86 architecture. Further details about how our evaluations went through will be mentioned in section 3 and 4. Here, we briefly go over two concepts that needs to be explained before moving on to details.

\subsection{Data race\label{sec:data_race}}

A {\it{data race}} is a situation where multiple threads try to access shared data at once, and the result depends on the execution order of the threads. For example, consider the following code:

\begin{lstlisting}[frame=tb, xleftmargin=2em, framexleftmargin=1.5em, numbers=left]
int a = 0;
int foo() {
    a++;
}
\end{lstlisting}

Suppose two threads call the \verb|foo| function at once. If the threads do not run concurrently, then the value of \verb|a| will be 2. However, this is not always the case when there is concurrency, because \verb|a++| actually consists of three steps: reading the value, incrementing the value, and writing back the value. The following sequence of actions may occur as a result: \begin{enumerate}
    \item Thread 1 reads the value of \verb|a|, which is 0.
    \item Thread 2 reads the value of \verb|a|, which is 0.
    \item Thread 1 increments its value, making it 1.
    \item Thread 2 increments its value, making it 1.
    \item Thread 1 writes 1 into the value of \verb|a|.
    \item Thread 2 writes 1 into the value of \verb|a|.
\end{enumerate}

We use locks to prevent this situation. When a thread tries to acquire a lock, but another thread owns the lock, then it must wait before the other thread releases the lock. For example, in the modified code below, two threads would wait for each other to read the value of verb|a|, increment it, and write it back. This keeps the global variable from being accessed twice at the same time.

\begin{lstlisting}[frame=tb, xleftmargin=2em, framexleftmargin=1.5em, numbers=left]
int a = 0;
int foo() {
    pthread_mutex_lock(&mutex_lock);
    a++;
    pthread_mutex_unlock(&mutex_lock);
}
\end{lstlisting}

\subsection{Machine instructions}
In compiled languages like \textit{C/C++}, each line of code is translated into multiple machine instructions. For example, consider \textbf{\textit{a++}} at third line of code snippet in \ref{sec:data_race}. It is adding one to integer variable \textbf{\textit{a}}. Although it is a single line of code in \textit{C}, a machine equipped with Intel x86 processors translates it into following instructions:
\begin{itemize}
	\item[] \textbf{ld} \,\,\, \$r1, \textit{a}
	\item[] \textbf{add} \$r1, \$r1, 1
	\item[] \textbf{st} \,\,\, \$r1, \textit{a}
\end{itemize}

It first loads the value of \textbf{\textit{a}} from memory into register \textit{\$r1}. Then, add one to the register value. Finally, it stores the result back to the memory location of \textbf{\textit{a}}. Note that these machine instructions are the commit point for each computation within a machine. It means that we can ensure there is no thread interleaving while executing each instruction. In fact, this is the reason why we are safe to count the number of instructions to measure performance of target program independent from other programs being executed.

\subsection{Critical section}

\subsection{Genetic algorithm}

\section{Experiment design}

\subsection{Environment setting}

MiniCS runs on Linux, requires Python (3.6 or higher) and Clang (v6.0.0 or higher). Note that \verb|gm.cpp| requires an external library, however one can also use pre-compiled binary file, \verb|gm|. To manually compile the source code, \verb|llvm-dev| must be installed into Clang~\cite{clang}. Our project code is open sourced in Github repository~\cite{miniCS}, and collaborators are as follows: Hyunsu Kim (\textit{hyunsukimsokcho}), Jaemin Yu (\textit{planetarynebula}), Doam Lee (\textit{ehdkacjswo}), Jaemin Choi (\textit{jh05013}), and Heeju Wi (\textit{bb0711}).


\subsection{Gene representation}

\subsection{Population generation}

As a prototype, we assume the following constraints:

\begin{enumerate}
    \item All scopes are explicitly given with curly braces.
    \item In the \verb|if|-branches, \verb|for|-branches, and \verb|while|-branches, there is no line break between the closing parenthesis and the opening brace. There is no line break between the braces and \verb|else|.
\end{enumerate}

Under these assumptions, we find a \textit{valid range} $[a, b]$, and modify the code so that a lock is held at the end of the $a$-th line and released at the end of the $b$-th line. A valid range to lock must satisfy the following four conditions:

\begin{enumerate}
    \item The range must be contained in a function.
    \item There must be a reference to a global variable inside the range. Otherwise there would be no need to place a lock.
    \item Every scope must either fully contain a range, or be fully contained in a range. This is to prevent a thread from holding its own lock or releasing a lock that it does not hold.
    \item There should be no \verb|return| statement in the range. This is to prevent a thread from finishing while holding a lock.
\end{enumerate}

For example, consider the following code:


\begin{lstlisting}[frame=tb, xleftmargin=2em, framexleftmargin=1.5em, numbers=left]
int arr[100];
int bar(int n) {
    int x = 0;
    for (int i = 0; i <= n; ++i) {
        arr[i] = x;
        x+= i;
    }
    return arr[n];
}
\end{lstlisting}

A range $[4, 5]$ is valid; in this case, we hold and release a lock right before and after executing \verb|arr[i] = x|. On the other hand, $[1, 9]$ is invalid because it is not contained in a function. $[5, 6]$ is invalid because there are no global variables. $[2, 5]$ is invalid because the scope inside the \verb|for| loop neither contains $[2, 5]$ nor is contained in $[2, 5]$. Finally, $[7, 8]$ is invalid because it contains \verb|return|.

We use Clang AST to detect the references to global variables. An expression in a function refers to a global variable if there is a \verb|DeclRefExpr| node whose reference variable has a global storage:

\begin{lstlisting}[frame=tb, xleftmargin=2em, framexleftmargin=1.5em, numbers=left]
static StatementMatcher global_match =
  declRefExpr(
    to(
      varDecl(
        hasGlobalStorage()
      ).bind("globalVar")
    ),
    hasAncestor(
      functionDecl().bind("function")
    )
  ).bind("globalRef");
  \end{lstlisting}

Next, a scope in a function is detected by a \verb|CoumpoundStmt| node:

\begin{lstlisting}[frame=tb, xleftmargin=2em, framexleftmargin=1.5em, numbers=left]
static StatementMatcher scope_match =
  compoundStmt(
    hasAncestor(
      functionDecl().bind("function")
    )
  ).bind("scope");
\end{lstlisting}

Finally, a return statement in a function is detected by a \verb|ReturnStmt| node:

\begin{lstlisting}[frame=tb, xleftmargin=2em, framexleftmargin=1.5em, numbers=left]
static StatementMatcher return_match =
  returnStmt(
    hasAncestor(
      functionDecl().bind("function")
    )
  ).bind("return");
\end{lstlisting}

After finding all references to global variables, all return statements, and all scopes, we output all possible valid ranges in the program. To generate a population, we choose zero or more valid ranges such that no two ranges with the same kind of lock do not intersect.

\subsection{Mutation}

\subsection{Crossover}

\subsection{Static analyzer}
What does it mean?

\subsection{Fitness evaluation}
Since our goal is to build code "without data race" and has "minimum lock interval", we have two objective for fitness evaluation. 

\subsubsection{Data race free}
First objective is number of "racing sets". The term "racing set" means the set of two lines that have data race. Since we need data race free code, we need to minimize the number of racing sets to 0. And this number can be obtained by data race detector, ThreadSanitizer. It's part of Clang that can detect data race for given code. The detection report will be offered like the sample below:

\begin{lstlisting}[frame=tb, xleftmargin=2em, framexleftmargin=1.5em, numbers=left]
WARNING: ThreadSanitizer: data race
  Write of size 4 at 0x7fe3c3075190:
    #0 bar1() simple_stack2.cc:16
    #1 Thread1(void*) simple_stack2.cc:34

  Previous read of size 4 at 0x7fe3c3075190:
    #0 bar2() simple_stack2.cc:29
    #1 main simple_stack2.cc:41
\end{lstlisting}

This report gives us four racing sets $(16, 29)$, $(16, 41)$, $(34, 29)$, $(34, 41)$ that each $(a, b)$ means $a$-th line and $b$-th line have data race. Since the detection result of ThreadSanitizer depends on execution order of threads, we execute ThreadSanitizer for every possible execution orders (for $n$ threads, execute for $n!$ times) and count the number of distinct racing sets.
 
\subsubsection{Lock interval}
Second objective is number of machine instructions between locks(including themselves). Lock interval is 

\section{Evaluation}


\section{Conclusion}
As a result,
Slowdown of the program is occured by two reasons
\begin{enumerate}
    \item ThreadSanitizer should be executed for every possible execution order of threads.
    \item GDB runs the program step by step between locks.
\end{enumerate}
To resolve this problem, 
Possible improvements (e.g. Coevolution with Thread schedule with Qemu emulator)


\end{document}
